%!TEX TS-program = xelatex
%!TEX encoding = UTF-8 Unicode
% Awesome CV LaTeX Template for Cover Letter
%
% This template has been downloaded from:
% https://github.com/posquit0/Awesome-CV
%
% Authors:
% Claud D. Park <posquit0.bj@gmail.com>
% Lars Richter <mail@ayeks.de>
%
% Template license:
% CC BY-SA 4.0 (https://creativecommons.org/licenses/by-sa/4.0/)
%


%-------------------------------------------------------------------------------
% CONFIGURATIONS
%-------------------------------------------------------------------------------
% A4 paper size by default, use 'letterpaper' for US letter
\documentclass[11pt, a4paper]{awesome-cv}

% Configure page margins with geometry
\geometry{left=1.4cm, top=.8cm, right=1.4cm, bottom=1.8cm, footskip=.5cm}

% Color for highlights
% Awesome Colors: awesome-emerald, awesome-skyblue, awesome-red, awesome-pink, awesome-orange
%                 awesome-nephritis, awesome-concrete, awesome-darknight
\colorlet{awesome}{awesome-skyblue}
% Uncomment if you would like to specify your own color
% \definecolor{awesome}{HTML}{CA63A8}

% Colors for text
% Uncomment if you would like to specify your own color
% \definecolor{darktext}{HTML}{414141}
% \definecolor{text}{HTML}{333333}
% \definecolor{graytext}{HTML}{5D5D5D}
% \definecolor{lighttext}{HTML}{999999}
% \definecolor{sectiondivider}{HTML}{5D5D5D}

% Set false if you don't want to highlight section with awesome color
\setbool{acvSectionColorHighlight}{true}

% If you would like to change the social information separator from a pipe (|) to something else
\renewcommand{\acvHeaderSocialSep}{\quad\textbar\quad}


%-------------------------------------------------------------------------------
%	PERSONAL INFORMATION
%	Comment any of the lines below if they are not required
%-------------------------------------------------------------------------------
% Available options: circle|rectangle,edge/noedge,left/right
\photo[rectangle,noedge,left]{./cv/photo.jpg}
\name{Jonas}{Wiedenmann}
\position{IT-Projektleiter, Softwareentwickler}
% \address{235, World Cup buk-ro, Mapo-gu, Seoul, 03936, Republic of Korea}

\mobile{+49 176 83512335}
\email{jonas-wiedenmann@outlook.de}
\github{jwiedenmann}
\linkedin{jonas-wiedenmann-622848340}{Jonas Wiedenmann}

% \quote{``Be the change that you want to see in the world."}


%-------------------------------------------------------------------------------
%	LETTER INFORMATION
%	All of the below lines must be filled out
%-------------------------------------------------------------------------------
% The company being applied to
\recipient
  {Müller Holding GmbH \& Co. KG - Personalabteilung}
  {Müller Holding GmbH \& Co. KG \\Albstraße 92\\89081 Ulm-Jungingen}
% The date on the letter, default is the date of compilation
\letterdate{22.03.2025}
% The title of the letter
\lettertitle{Bewerbung zum Projektleiter / Projektmanager E-Commerce}
% How the letter is opened
\letteropening{Sehr geehrte Damen und Herren,}
% How the letter is closed
\letterclosing{Mit freundlichen Grüßen}
% Any enclosures with the letter
% \letterenclosure[Attached]{Curriculum Vitae}


%-------------------------------------------------------------------------------
\begin{document}

% Print the header with above personal information
% Give optional argument to change alignment(C: center, L: left, R: right)
\makecvheader[R]

% Print the footer with 3 arguments(<left>, <center>, <right>)
% Leave any of these blank if they are not needed
\makecvfooter
  {22.03.2025}
  {Jonas Wiedenmann~~~·~~~Anschreiben}
  {}

% Print the title with above letter information
\makelettertitle

%-------------------------------------------------------------------------------
%	LETTER CONTENT
%-------------------------------------------------------------------------------
\begin{cvletter}

  mit großem Interesse bewerbe ich mich für die Position als Projektleiter / Projektmanager E-Commerce an Ihrem Standort in Ulm. Die Verbindung von IT und digitalem Vertrieb finde ich spannend, besonders weil ich gerne an praxisnahen, nutzerorientierten Lösungen arbeite und Erfahrung in der Steuerung technischer Projekte mitbringe.

  Derzeit arbeite ich als Projektleiter in einem mittelständischen Unternehmen, in dem ich IT-Projekte von der Planung bis zur Einführung begleite. Dabei gehört es zu meinen Aufgaben, Anforderungen mit verschiedenen Fachbereichen abzustimmen, Umsetzungen zu koordinieren und Projektziele im Blick zu behalten. Mein technischer Hintergrund hilft mir, Entwicklungen realistisch einzuschätzen und verständlich zu vermitteln.
  
  Mein Einstieg ins Projektmanagement baut auf meiner mehrjährigen Erfahrung in der Softwareentwicklung auf, in der ich Anwendungen geplant und umgesetzt habe. Ergänzt wurde dieses praktische Wissen durch mein Informatikstudium im Bachelor, das mir ein solides technisches Fundament vermittelt hat. Dieses Verständnis hilft mir heute, Anforderungen und Umsetzung besser einzuordnen und unterschiedliche Perspektiven zusammenzubringen.
  
  Auch wenn ich inzwischen erste Erfahrungen in der Projektleitung gesammelt habe, sehe ich mich noch am Anfang meiner Entwicklung in diesem Bereich. Um mich hier gezielt weiterzuentwickeln, studiere ich aktuell berufsbegleitend IT-Projekt- und Prozessmanagement im Master. Das Studium liefert mir wertvolle Impulse zu Methoden, Strukturen und Rollen im Projektumfeld und hilft mir dabei, meine Arbeit noch strukturierter und zielgerichteter zu gestalten. Der geplante Wechsel des Arbeitgebers ist für mich ein wichtiger Schritt, um meine Kenntnisse in einem professionellen Umfeld weiter auszubauen, Verantwortung zu übernehmen und von eingespielten Prozessen sowie interdisziplinärer Zusammenarbeit zu lernen.
  
  Ich finde es spannend zu sehen, wie sich Ihr Unternehmen im E-Commerce weiterentwickelt, und sehe darin eine gute Gelegenheit, meine Kenntnisse einzubringen und weiter auszubauen. Gerne stelle ich mich Ihnen in einem persönlichen Gespräch vor und freue mich auf Ihre Rückmeldung.

\end{cvletter}


%-------------------------------------------------------------------------------
% Print the signature and enclosures with above letter information
\makeletterclosing

\end{document}
