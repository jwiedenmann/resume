%!TEX TS-program = xelatex
%!TEX encoding = UTF-8 Unicode
% Awesome CV LaTeX Template for Cover Letter
%
% This template has been downloaded from:
% https://github.com/posquit0/Awesome-CV
%
% Authors:
% Claud D. Park <posquit0.bj@gmail.com>
% Lars Richter <mail@ayeks.de>
%
% Template license:
% CC BY-SA 4.0 (https://creativecommons.org/licenses/by-sa/4.0/)
%


%-------------------------------------------------------------------------------
% CONFIGURATIONS
%-------------------------------------------------------------------------------
% A4 paper size by default, use 'letterpaper' for US letter
\documentclass[11pt, a4paper]{awesome-cv}

% Configure page margins with geometry
\geometry{left=1.4cm, top=.8cm, right=1.4cm, bottom=1.8cm, footskip=.5cm}

% Color for highlights
% Awesome Colors: awesome-emerald, awesome-skyblue, awesome-red, awesome-pink, awesome-orange
%                 awesome-nephritis, awesome-concrete, awesome-darknight
\colorlet{awesome}{awesome-skyblue}
% Uncomment if you would like to specify your own color
% \definecolor{awesome}{HTML}{CA63A8}

% Colors for text
% Uncomment if you would like to specify your own color
% \definecolor{darktext}{HTML}{414141}
% \definecolor{text}{HTML}{333333}
% \definecolor{graytext}{HTML}{5D5D5D}
% \definecolor{lighttext}{HTML}{999999}
% \definecolor{sectiondivider}{HTML}{5D5D5D}

% Set false if you don't want to highlight section with awesome color
\setbool{acvSectionColorHighlight}{true}

% If you would like to change the social information separator from a pipe (|) to something else
\renewcommand{\acvHeaderSocialSep}{\quad\textbar\quad}


%-------------------------------------------------------------------------------
%	PERSONAL INFORMATION
%	Comment any of the lines below if they are not required
%-------------------------------------------------------------------------------
% Available options: circle|rectangle,edge/noedge,left/right
\photo[rectangle,noedge,left]{./cv/photo.jpg}
\name{Jonas}{Wiedenmann}
\position{IT-Projektleiter, Softwareentwickler}
% \address{235, World Cup buk-ro, Mapo-gu, Seoul, 03936, Republic of Korea}

\mobile{+49 176 83512335}
\email{jonas-wiedenmann@outlook.de}
\github{jwiedenmann}
\linkedin{jonas-wiedenmann-622848340}{Jonas Wiedenmann}

% \quote{``Be the change that you want to see in the world."}


%-------------------------------------------------------------------------------
%	LETTER INFORMATION
%	All of the below lines must be filled out
%-------------------------------------------------------------------------------
% The company being applied to
\recipient
  {adesso SE - Personalabteilung}
  {adesso SE\\Adessoplatz 1\\44269 Dortmund}
% The date on the letter, default is the date of compilation
\letterdate{08.03.2025}
% The title of the letter
\lettertitle{Bewerbung zum Teamleiter Softwareentwicklung}
% How the letter is opened
\letteropening{Sehr geehrte Damen und Herren,}
% How the letter is closed
\letterclosing{Mit freundlichen Grüßen}
% Any enclosures with the letter
% \letterenclosure[Attached]{Curriculum Vitae}


%-------------------------------------------------------------------------------
\begin{document}

% Print the header with above personal information
% Give optional argument to change alignment(C: center, L: left, R: right)
\makecvheader[R]

% Print the footer with 3 arguments(<left>, <center>, <right>)
% Leave any of these blank if they are not needed
\makecvfooter
  {08.03.2025}
  {Jonas Wiedenmann~~~·~~~Anschreiben}
  {}

% Print the title with above letter information
\makelettertitle

%-------------------------------------------------------------------------------
%	LETTER CONTENT
%-------------------------------------------------------------------------------
\begin{cvletter}

  mit großem Interesse bewerbe ich mich für die Position als Teamleiter Softwareentwicklung in Ihrem Unternehmen. Die Möglichkeit, Verantwortung für IT-Projekte zu übernehmen, Teams weiterzuentwickeln und innovative Softwarelösungen zu gestalten, reizt mich besonders. Die Verbindung von technischer Expertise mit organisatorischen und strategischen Aufgaben entspricht genau meinem beruflichen Fokus.
  
  Aktuell arbeite ich als Projektleiter in der Softwareentwicklung und leite eigenverantwortlich IT-Projekte. Neben meiner beruflichen Tätigkeit absolviere ich ein berufsbegleitendes Masterstudium im Studiengang "IT-Projekt- und Prozessmanagement", um meine Kompetenzen im Bereich Projektmanagement und Führung gezielt weiterzuentwickeln. Durch mein duales Bachelorstudium in Informatik habe ich fundierte Kenntnisse in der Softwareentwicklung erworben. Seit Abschluss meines Studiums konnte ich wertvolle Erfahrungen im Management von IT-Projekten sammeln und meine Fähigkeiten in der Projektsteuerung weiter ausbauen. In meinem aktuellen Unternehmen liegt der Fokus auf der Entwicklung mit C\# und .NET. Dabei habe ich umfassende Erfahrung sowohl in der Entwicklung von Client-Anwendungen mit WPF als auch in der Web-Entwicklung mit ASP.NET Core gesammelt. Diese Kombination aus technischer Expertise und Projektverantwortung ermöglicht es mir, Softwareprojekte ganzheitlich zu betrachten und strategisch zu steuern.
  
  Durch meine Erfahrung in der technischen Umsetzung von Softwareprojekten sowie meine Führungskompetenz bringe ich die notwendigen Qualifikationen mit, um Teams erfolgreich zu leiten. Mein strukturiertes und zielorientiertes Vorgehen, kombiniert mit Kommunikationsstärke und Entscheidungsfreude, erlaubt es mir, sowohl in der Führung von Mitarbeitenden als auch in der Projektsteuerung nachhaltig Mehrwert zu schaffen.
  
  Besonders an Ihrem Unternehmen hat mich angesprochen, dass Sie die Leidenschaft für IT in den Mittelpunkt stellen, da mich genau diese Begeisterung auch antreibt. Meine Faszination für IT begann schon früh: Schon immer wollte ich verstehen, wie moderne Technologien funktionieren und vor allem, wie man sie entwickelt. Dieses Interesse hat mich zunächst zu meiner Ausbildung als Fachinformatiker geführt, dann zu meinem dualen Bachelorstudium in Informatik und schließlich zu meinem Masterstudium in IT-Projekt- und Prozessmanagement. In all diesen Stationen habe ich mein Wissen stetig vertieft und mich weiterentwickelt. Doch eines ist für mich essenziell: meine Leidenschaft für die Softwareentwicklung und Technik zu bewahren. Ich bin überzeugt, dass fundierte technische Kenntnisse und die enge Verbindung zur praktischen Umsetzung essenziell für den nachhaltigen Erfolg in IT-Projekten sind.
  
  Gerne möchte ich Sie in einem persönlichen Gespräch davon überzeugen, dass ich die richtige Person für diese Position bin. Ich freue mich auf Ihre Rückmeldung.

\end{cvletter}


%-------------------------------------------------------------------------------
% Print the signature and enclosures with above letter information
\makeletterclosing

\end{document}
